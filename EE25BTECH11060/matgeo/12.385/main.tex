\let\negmedspace\undefined
\let\negthickspace\undefined
\documentclass[journal]{IEEEtran}
\usepackage[a5paper, margin=10mm, onecolumn]{geometry}
%\usepackage{lmodern} % Ensure lmodern is loaded for pdflatex
\usepackage{tfrupee} % Include tfrupee package

\setlength{\headheight}{1cm} % Set the height of the header box
\setlength{\headsep}{0mm}     % Set the distance between the header box and the top of the text

\usepackage{gvv-book}
\usepackage{gvv}
\usepackage{cite}
\usepackage{amsmath,amssymb,amsfonts,amsthm}
\usepackage{algorithmic}
\usepackage{graphicx}
\usepackage{textcomp}
\usepackage{xcolor}
\usepackage{txfonts}
\usepackage{listings}
\usepackage{enumitem}
\usepackage{mathtools}
\usepackage{gensymb}
\usepackage{comment}
\usepackage[breaklinks=true]{hyperref}
\usepackage{tkz-euclide} 
\usepackage{listings}
% \usepackage{gvv}                                        
\def\inputGnumericTable{}                                 
\usepackage[latin1]{inputenc}                                
\usepackage{color}                                            
\usepackage{array}                                            
\usepackage{longtable}                                       
\usepackage{calc}                                             
\usepackage{multirow}                                         
\usepackage{hhline}                                           
\usepackage{ifthen}                                           
\usepackage{lscape}
\begin{document}

\bibliographystyle{IEEEtran}
\vspace{3cm}

\title{12.385}
\author{EE25BTECH11060 - V.Namaswi}
% \maketitle
% \newpage
% \bigskip
{\let\newpage\relax\maketitle}
\renewcommand{\thefigure}{\theenumi}
\renewcommand{\thetable}{\theenumi}
\setlength{\intextsep}{10pt} % Space between text and floats
\textbf{Question}\\
Given the following matrix equation 
 \[
A_{m \times n} X_{x \times 1} = b_{m \times 1}
\]  the nature of this system of equations is \\
\begin{multicols}{2}
    \begin{enumerate}
\item  over determined if $m>n$
 \item  under determined if $m<n$
 \item  even determined if $m=n$
 \item  determined by rank of the matrix 
    \end{enumerate}
\end{multicols}
\textbf{Solution}\\
Given the system of equations 
\[
A_{m \times n} X_{n \times 1} = b_{m \times 1},
\] 
As m determine number of equations and n number of unknowns\\ 
  
\begin{itemize}
    \item If $m > n$, there are more equations than unknowns   \textit{over-determined}.
    \item If $m < n$, there are fewer equations than unknowns   \textit{under-determined}.
    \item If $m = n$, the system is \textit{even-determined} (square system).
    \end{itemize}
However, just knowing $m$ and $n$ does \textbf{not} guarantee a solution, because some equations may be \textbf{linearly dependent}.

\begin{itemize}
    \item The rank of $A$ gives the number of \textbf{independent equations}.
    \item For a square system ($m=n$), a unique solution exists only if $\text{rank}(A) = n$.
    \item If $\text{rank}(A) < n$, the system may have \textbf{no solution or infinitely many solutions}.
    \item For non-square systems ($m \neq n$), the rank still determines if a solution exists and how many solutions are possible.
\end{itemize}

\textbf{Conclusion:}  

The actual nature of the system depends on the \textbf{rank of the matrix $A$}, not just on the number of equations and unknowns.  
\end{document}